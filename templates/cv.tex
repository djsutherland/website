
%% This file is very loosely based on a template by Xavier Danaoux
%% (https://github.com/xdanaux/moderncv),
%% who released it under the following license:
%
% Template copyright 2006-2011 Xavier Danaux (xdanaux@gmail.com).
%
% This work may be distributed and/or modified under the
% conditions of the LaTeX Project Public License version 1.3c,
% available at http://www.latex-project.org/lppl/.

\documentclass[11pt,sans]{moderncv}

\usepackage[utf8]{inputenc}
\usepackage[american]{babel}
\usepackage{csquotes}

\moderncvtheme[grey]{classic}

\usepackage{paralist}
\usepackage{ifthen}

% adjust the page margins
\usepackage[scale=0.8]{geometry}
\setlength{\hintscolumnwidth}{2.2cm} % width of the column with the dates

\usepackage[maxbibnames=100,doi=false,url=false,sorting=none,style=authortitle,dashed=false]{biblatex}
\DeclareNameAlias{author}{given-family}

\DeclareFieldFormat[misc]{title}{\mkbibquote{#1\isdot}}
\renewbibmacro{in:}{}

\definecolor{student}{RGB}{200,100,0}

\newcommand{\bibeqcon}{\textcolor{blue}{\footnotemark[1]}}
\renewcommand*{\mkbibnamegiven}[1]{%
  \ifitemannotation{me}%
    {#1}%{\textbf{#1}}%
    {\ifitemannotation{mystudent}{\textcolor{student}{#1}}{#1}}}
\renewcommand*{\mkbibnamefamily}[1]{%
  \ifitemannotation{me}%
    {#1}%{\textbf{#1}}%
    {\ifitemannotation{mystudent}{\textcolor{student}{#1}}{#1}}%
  \ifitemannotation{equal}
    {\bibeqcon}
    {}%
}

% ignore pubstate
\DeclareSourcemap{
    \maps[datatype=bibtex]{
        \map[overwrite]{
            \step[fieldset=pubstate,fieldvalue={}]
        }
    }
}




  % ["Dissertations", ["phd-thesis", "ba-thesis"]],


  \defbibfilter{sec{{ loop.index -}} }{
    
      keyword={{ sec }}
       or 
    
  }



\addbibresource{biblio-cv.bib}

\newcommand{\httpurl}[1]{\href{http://#1}{\nolinkurl{#1}}}
\newcommand{\httpsurl}[1]{\href{https://#1}{\nolinkurl{#1}}}

% personal data
\firstname{Danica J.}
\familyname{Sutherland}

\nopagenumbers{}

%----------------------------------------------------------------------------------
%            content
%----------------------------------------------------------------------------------
\begin{document}
\maketitle
\vspace*{-15mm}

\section{Contact Information}
\newcommand{\phoneno}[3]{#1$\,\cdot\,$#2$\,\cdot\,$#3}
\begin{tabular}{@{}p{3.13in}p{4in}}
Computer Science  & \textit{email}: \texttt{dsuth@cs.ubc.ca} \\
Vancouver Campus  & \textit{web}: \httpsurl{cs.ubc.ca/~dsuth/} \\
ICICS/CS Building 201-2366 Main Mall &\\
Vancouver, BC Canada $\;\;$ V6T 1Z4 &
\end{tabular}

\section{Research interests}
Machine learning,
particularly kernel methods, their integration with deep learning,
and representation learning more broadly.
Learning and testing on sets and distributions,
including two-sample tests,
generative models, density estimation, and distribution regression.
Nonparametric statistics, statistical learning theory.

\section{Academic Positions and Education}
\cventry{2021 -- \phantom{2}}{Assistant Professor}{Computer Science Department, University of British Columbia}{}{}{}
\cventry{2021 -- \phantom{2}}{Canada CIFAR AI Chair}{Alberta Machine Intelligence Institute (Amii)}{}{}{}
\cventry{2019 -- 2020}{Research Assistant Professor}{Toyota Technological Institute at Chicago}{}{}{}
\cventry{2016 -- 2019}{Research Associate}{Gatsby Computational Neuroscience Unit, University College London}{}{}{Postdoctoral position with Arthur Gretton.}
\cventry{2011 -- 2016}{Ph.D., Computer Science}{Carnegie Mellon University}{}{}{%
  \emph{Thesis Title}: Scalable, Flexible, and Active Learning on Distributions. \\
  \emph{Committee}: Jeff Schneider (chair), Barnab\'as P\'oczos, Maria-Florina Balcan, Arthur Gretton.\\
  Included an M.S.\ obtained in 2015. % Unofficial GPA: 3.96 / 4.
}
% \cventry{2015}{M.S., Computer Science}{Carnegie Mellon University}{}{}{}
\cventry{2007 -- 2011}{B.A., Computer Science}{Swarthmore College}{}{with high honors}{%
    Minors in Linguistics (with high honors) and Mathematics \& Statistics. \\ % GPA: 3.93 / 4. \\
    \emph{Thesis Title}: Integrating Human Knowledge into a Relational Learning System.
}


% \section{Honors and Awards}
% \cvline{2014 -- 2016}{Sandia Campus Executive Program fellowship. (Renewed in 2015.)}
% \cvline{2013}{National Science Foundation Graduate Research Fellowship Program: Honorable Mention.}
% \cvline{2011}{Ivy Award for ``the senior man outstanding in leadership, scholarship, and contributions to the college community'' by Swarthmore faculty vote.}
% \cvline{2011}{Elected Phi Beta Kappa.}
% \cvline{2011}{Drew Pearson Prize for excellence in journalism.}
% \cvline{2009--10}{Associated Collegiate Press Online Pacemaker Award for the Swarthmore \emph{Daily Gazette}.}


% \section{Research and Academic Experience}
% \cventry{2013 -- 2016}{XDATA workshops}{DARPA}{}{}{%
%     Analyzed various datasets with teams from across academia and industry as a testbed for development of open-source data-analytic software libraries.
%     Developed a Python library, \texttt{skl-groups}, for machine learning on distributions.
%     Participated in development of a financial analysis application in use at a federal agency.
%     Led a small team of CMU participants and managed collaborations.
% }
% \cventry{2011 -- 2016}{Ph.D.\ research}{Carnegie Mellon University}{}{}{%
%     Research in machine learning with Jeff Schneider. %; frequent collaboration with Barnab\'as P\'oczos.
%     Particular focus on machine learning on samples from distributions and on active learning problems.
%     % In addition to work represented by the publications below,
%     Additional unpublished empirical work related to learning on distributions in the analysis of financial anomalies, fusion reactor behavior, web browsing traffic, shipping behavior, terrorist activities, and Twitter language use.
% }
% \cventry{2011}{Linguistics senior honors study}{Swarthmore College}{}{}{%
%     Studied the phonotactics of Chaha, including computational approaches, for Colleen Fitzgerald.
% }
% \cventry{2010}{REU in machine learning}{University of Oklahoma}{}{}{%
%     Improved and added human interaction to a relational concept learning system with Andrew Fagg.
% }
% \cventry{2009}{Howard Hughes Medical Institute fellowship}{Swarthmore College}{}{}{%
%     Worked on natural language processing and medical information extraction with Rich Wicentowski.
% }
% \cventry{2009}{Directed independent study project}{Pitzer in Nepal}{}{}{%
%     Examined the interaction of language use and pedagogical techniques in rural Nepali schools.
% }

\section{Students}
\begin{tabular}{l@{\hskip .2in}l@{\hskip .2in}l@{\hskip .2in}l@{\hskip .2in}l}
    2020 -- & Ph.D. & UBC CS & Wonho Bae &\\
    2020 -- & Ph.D. & UBC CS & Yi (Joshua) Ren &\\
    2021 -- & Ph.D. & UBC CS & Hamed Shirzad &\\
    2020 -- & M.Sc. & UBC CS & Milad Jalali &\\
    2020 -- & M.Sc. & UBC CS & Namrata Deka &\\
    2021 -- & M.Sc. & UBC CS & \multicolumn{2}{l}{Mohamad Amin Mohamadi} \\
    2022 -- & M.Sc. (course) & UBC CS & Arsh Jhaj \\
    %2020 -- & Ph.D. & UChicago Stat & Lijia Zhou & (Informal co-supervision) \\
    %2020 & Ph.D. & TTI-Chicago& Akilesh Tangella & (Informal co-supervision) \\
\end{tabular}

\subsection{Thesis Committees}
\begin{tabular}{l@{\hskip .2in}l@{\hskip .2in}l@{\hskip .2in}l}
    2020 & Iryna Korshunova & Doctoral & Ghent University \\
    2020 & Tong (Gerry) Che & Masters & University of Cambridge
\end{tabular}

\section{Funding}
\begin{tabular}{l@{\hskip .2in}l@{\hskip .2in}l@{\hskip .2in}l@{\hskip .2in}l}
  2021 -- 2025 & NSERC & Discovery Grant & \$157,000 & PI \\
  2021 -- 2025 & CIFAR & Canada CIFAR AI Chair & \$391,666 & PI \\
\end{tabular}

\clearpage  % TEMP

\section{Publications}
\renewcommand{\thefootnote}{\fnsymbol{footnote}\fnsymbol{footnote}}
\setlength\bibitemsep{0.3\baselineskip}
Below, \bibeqcon{} denotes equal contribution,
and \textcolor{student}{this colour} denotes one of my formal advisees.
\nocite{*}



\printbibliography[filter=sec{{ loop.index }},title={ {{- sec_name -}} },heading=subbibnumbered]{}



% \clearpage  % TEMP
\section{Invited Talks}
\newcommand{\talk}[3]{% date, name, venue
  \item[#1]%
    \ifthenelse{\equal{#2}{}}{}{\textit{#2}\\}
    #3
}
\begin{itemize}


    
\talk{ {{- "{:%b %Y}".format(talk.date) -}} }%
     { {{- talk.title | maybe_punc(".") -}} }%
     {
        
        {{- venue.name -}}
        {{- venue.short | maybe_wrap(" (", ")") -}}
        
          
          {{- "" }} (workshop at {{ conf.long_article | maybe_wrap("", " ") -}} {{- conf.name -}} {{- conf.short | maybe_wrap(", ", "") }})
        
        .
        {{- talk.note | maybe_wrap(" ", "") -}}
     }



\end{itemize}

% \clearpage % TEMP
\section{Teaching}

\cventry{Fall 2022}{Instructor}{CPSC 532D Topics in AI: Modern Statistical Learning Theory}{UBC}{}{Graduate-level overview of statistical learning theory; 24 students.}

\cventry{Spring 2022}{Instructor}{CPSC 532S Topics in AI: Modern Statistical Learning Theory}{UBC}{}{Graduate-level overview of statistical learning theory; 23 students.}

\cventry{Fall 2021}{Instructor}{CPSC 340 Machine Learning and Data Mining}{UBC}{}{Third-year undergraduate introduction to machine learning; 228 students between two sections, co-taught with Mike Gelbart.}

\cventry{}{Guest lectures, one-day courses, and similar}{}{}{}{%
  \begin{compactitem}
  \item
      ``Modern Kernel Methods in Machine Learning.''
      October 2022.\\
      ETICS summer school, Corsica.
      (6 hours, c.\ 55 students.)
    \item
      ``Kernel Methods: From Basics to Modern Applications.''
      January 2021.\\
      Data Science Summer School,
      \'Ecole Polytechnique, Paris.
      (3 hours, twice, c.\ 12 students each.)
    \item
      ``Learning With Positive Definite Kernels: Theory, Algorithms, and Applications.''
      June 2019.\\
      With Bharath Sriperumbudur.
      Data Science Summer School,
      \'Ecole Polytechnique, Paris.
      \\(6 hours, twice, c.\ 25 students each.)
    \item
      ``Generative Adversarial Networks.''
      June 2019.\\
      Machine Learning Crash Course,
      University of Genova, Italy.
      (1.5 hours, c.\ 100 students.)
    \item
      ``New Kernel Distances for Better Deep Generative Models.''
      December 2018.\\
      Advanced Topics in Machine Learning,
      University College London.
      (1 hour, c.\ 20 students.)
    \item
      ``What Is Machine Learning?''
      April 2016.\\
      Capstone Course,
      Jackson Institute for Global Affairs, Yale University.
      (1 hour, c.\ 15 students.)
    \item
      ``What Is Machine Learning?''
      December 2014.\\
      Capstone Course,
      Jackson Institute for Global Affairs, Yale University.
      (1 hour, c.\ 15 students.)
  \end{compactitem}
}

\cventry{Spring 2014}{Teaching Assistant}{Algorithms in the Real World}{CMU}{Guy Blelloch/Anupam Gupta}{%
    % Ph.D.-level course on algorithms with real-world applications.
    % (Guy Blelloch and Anupam Gupta)
}
\cventry{Fall 2013}{Teaching Assistant}{Machine Learning}{CMU}{Alex Smola/Geoff Gordon}{%
    % Introductory Ph.D.-level course in machine learning.
    % (Alex Smola and Geoff Gordon)
}
\cventry{Summer 2011}{Teaching Assistant and Residential Mentor}{Summer Science Program}{\httpurl{ssp.org}}{}{%
    Intense five-week program for high schoolers from around the world,
    who learned programming, vector calculus, and astronomy
    to determine an asteroid's orbit from their own observations.
}
% \cventry{2009 -- 2011}{Editor-in-Chief}{The Daily Gazette}{Swarthmore}{}{%
%     Supervised staff in writing and editing news stories,
%     as well as managing all newspaper operations.
% }
% \cventry{2008 -- 2011}{Lead Web Developer}{The Daily Gazette}{Swarthmore}{}{%
%     Led small teams in developing an award-winning newspaper site and a campus announcement site.
% }
% \cventry{2008 -- 2011}{System Administrator}{Swarthmore College Computing Society}{}{}{}
% \cventry{2009 -- 2011}{IT Associate}{Swarthmore College IT Services}{}{}{}

\section{Service}
\subsection{Scholarly activities}
\cventry{2019 --}{Area chair or equivalent}{NeurIPS 2020--22, ICML 2022, AISTATS 2020--23, AAAI 2021--22}{}{}{}
\cventry{2015 --}{Program committee or equivalent}{ICML 2016--21; NeurIPS 2015, 2017--19; ICLR 2018--20; AISTATS 2019; AAAI 2018}{}{}{NeurIPS 2018: top 216 (of 3,045) reviewers. NeurIPS 2019: Top Reviewer. ICML 2018: Outstanding Reviewer. ICML 2019: Best Reviewer.}
\cventry{2014 --}{Reviewer}{JMLR, IEEE TSP, IEEE T-PAMI, Bernoulli, MLJ, COLT, SoCG, IJCAI, ECML-PKDD, UAI, Comptes rendus}{}{}{}
% \cventry{2017}{Session chair}{ICML}{}{}{}

\subsection{Departmental service}
\cventry{2022}{Faculty recruiting committee: CRC Tier 2 in Quantum Computing}{UBC CS}{}{}{}
\cventry{2021}{Graduate Recruitment and Admissions Committee}{UBC CS}{}{}{}
\cventry{2016 -- 2019}{External seminar organizer}{Gatsby}{}{}{}
\cventry{2013}{Immigration Course organizer}{CMU}{}{}{}

\subsection{Other}
\cventry{2018 --}{Volunteer organizer}{Queer in AI}{}{}{Core organizer since 2022. Activities including D\&I advocacy to conferences, co-organizing AAAI 2022 social and mentoring activities, serving on a hiring committee, initiating a program of informal feedback and mentoring for faculty candidates, and more.}
\cventry{2015 --}{Top 100 all-time contributor}{Cross Validated}{\nolinkurl{stats.stackexchange.com}}{}{}
\cventry{2018 -- 2020}{Core member}{}{\httpsurl{conda-forge.org}}{scientific software packaging ecosystem}{(Emeritus since 2020.)}


% \section{Graduate Coursework}
% \begin{tabular}{l@{\hskip .1in}l@{\hskip .1in}l@{\hskip .2in}l@{\hskip .2in}l}
%   CMU & F2013 \hfill & Deep Learning  \hfill & B Raj \hfill     & A   \\
%   CMU & S2013 & Optimizing Compilers for Modern Architectures & T Mowry & A \\
%   CMU & F2012 & Optimization            & G Gordon, R Tibshirani & A+  \\
%   CMU & F2012 & Intermediate Statistics & L Wasserman            & A+  \\
%   CMU & S2012 & Graduate Algorithms     & M Blum                 & A--  \\
%   CMU & S2012 & Semantics of Programming Languages & S Brookes   & A   \\
%   CMU & F2011 & Machine Learning        & E Xing                 & A+  \\
%   CMU & F2011 & Computational Models of Neural Systems & D Touretzky & A \\
%   UPenn & S2010 & Software Foundations  & B Pierce               & A+
% \end{tabular}

% \section{Other}
% \cvline{Programming}{Expert: Python scientific/deep learning ecosystem. Experienced: C/C++, web languages.}% Have extended scikit-learn, Caffe, Django, LLVM, Postgres, and others.}
% %\cvline{Software}{Standard Unix and Macintosh systems, Git, SVN, \LaTeX. System administration on Debian.}
% \cvline{Citizenship}{U.S.}

\begin{center}
\footnotesize \textit{Last update: 
{{ "{:%-d %B, %Y}".format(['templates/cv.tex', 'papers.yaml'] | last_edit_dt) }}

.}
\end{center}

\end{document}

